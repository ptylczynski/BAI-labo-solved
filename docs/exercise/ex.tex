\documentclass{article}
\usepackage[T1]{fontenc}
\usepackage{graphicx}
\usepackage{listings}
\usepackage{xcolor}
\usepackage{minted}
\usepackage{hyperref}
\hypersetup{
    colorlinks=true,
    linkcolor=blue,
    filecolor=magenta,      
    urlcolor=cyan,
}

\graphicspath{ {../img/} }

\begin{document}
    \title{Laboratorium Podstawy Aplikacji Internetowych \\
    Spring}
    \author{Piotr Tylczyński \\
    \texttt{piotr.tylczynski@ptl.cloud}}
    
    \begin{titlepage}
        \maketitle
    \end{titlepage}
    
    \tableofcontents
    \pagebreak
    
    \section{Wstęp}
    Celem tego laboratorium jest pokazanie działanie frameworku Spring w praktyce. Podczas zajęć wykonasz prosty serwer, który obsłuży API dla wypożyczalni samochodów. 
    
    Stworzone przez ciebie rozwiązanie będzie bardzo uproszczone w porównaniu do prawdziwych serwerów jakie obsługiwały by takie zadanie. Jednak nie oznacza to że będzie bezużyteczne. Nie licząc braku systemu autoryzacji, stworzysz serwer zgodnie z aktualnymi standardami produkcyjnymi.
    
    \section{Założenia projektu}
        Tworzymy prosty serwer backendowy służący do obsługi wyporzyczalni samochodów. Chcielibyśmy móc na bierząco śledzić bazę samochodową, oraz dodawać do niej nowe samochody. Dodatkowo, będziemy prowadzić spis klientów korzystających z naszych usług, tak aby móc śledzić, kto i kiedy coś wyporzyczył.
    
    \section{Inicjalizacja projektu}
        Całość tworzenia kodu rozpoczniemy od stworzenia projektu. W tym celu mamy dwa wyjścia. W pierwszym możemy ręcznie szukać odpowiednich modułów Springa w internecie i repozytoriach \emph{Mavena}. W drugim skorzystamy z rozwiązania Spring Initializr. Trzymając się założeń, że korzystamy z rowiązań wykorzystywanych w przemyśle skorzystamy z Initializr. 
        
        Initializr został stworzony w 2013 roku przez firmę VMware, Inc. i jest prostą aplikacją służącą do szybkiego tworzenia baz dla projektów Springa. Możemy w nim wybrać interesujące nas moduły a aplikacja sama zajmie się wyszukaniem ich w repozytoriach Maven i dodaniem odpowiednich wpisów w \emph{pom.xml} jaki i stworzeniem odpowiednich plików w samym projekcie. 
        
        Osoby chętne mogą same skorzystać z Initializr i przejść przez kreator. W tym celu należy wskazać moduły:
        \begin{itemize}
            \item Lombok
            \item Spring Web
            \item Spring Data JPA
            \item H2 Database
        \end{itemize}
        Zachęcam jednak do pobrania gotowego prekonfigurowanego repozytorium z linku na końcu skryptu. Pomoże to w uniknięciu prostych błędów i rozbierzności w samej konfiguracji projektu. 
        
        Ważnym elementem do rozważenia na tym etapie jest struktura projektu. Spring nie narzuca nic konkretnego, jednak skorzystamy z podejścia jedna funkcjinalność, jeden folder. Pomoże to nam w przyszłości szybko nawigować się po projekcie. Szczególnie jeśli kiedyś zdecydujemy się na jego rozwijanie. Zastanówmy się teraz jakie funkcjonalności mamy do zaimplementowania. Na pewno jest to element zajmujący się klientami, oraz element zajmujący się samochodami. Z tego powodu stworzymy dwa foldery. Jeden dedykowany dla kodu obsługującego logikę kilentów i jeden dla logiki aut.
        
        \begin{figure}[h]
            \centering
            \includegraphics{tree.png}
        \end{figure}
    
    \section{Modelowanie Bazy Danych}
        Najprostszym, jednak bardzo nie efektywnym sposobem będzie ręczne stworzenie schematu bazy danych przez podawanie odpowiednich komend SQL. Następnie do obsługi takiej bazy danych użylibyśmy biblioteki JDBC, tak jak było to pokazywane na zajęciach z Systemów Baz Danych podczas 5 semestru. Jednak jak wspomniałem na początku jest to podejście bardzo nie efektywne. Wyobraźmy sobie, że zamiast modelowania tylko dwóch elementów chcielibyśmy stworzyć ich dwadzieścia. Być może stworzenie bazy danych nie byłoby wyzwaniem, jednak stworzenie i zarządanie skryptami JDBC już tak. Dalej byłoby jeszcze gorzej, ponieważ bazę danych od czasu do czasu trzeba zmieniać, zgodnie ze zmieniającymi się wymaganiami projektu, a to niesie ze sobą potrzebe ręcznego przemodelowania bazy danych a to zmianę skryptów. A na końcu musimy pamiętać, że wiele baz danych posiada swoje charakterystyczne dialekty, co wcale nie ułatwia tworzenia agnostycznego kodu. 
        
        Z tego powodu skorzystamy z jednego z modułów, który sami dodaliśmy - Spring Data. Jest to moduł, który za pomocą standardu JPA - Java Persistence API - oraz projektu Hibernate pozwala na modelowanie bazy danych jako obiektów Javy. Hibernate to ORM - Object Relation Mapper. W ten sposób możemy całkowicie zapomnieć o pisaniu czegokolwiek samemu. Od teraz będziemy tworzyć obiekty POJO i odpowiednio je oznaczać a Hibernate zrobi wszystko za nas.
        
        \subsection{Konfigurowanie Bazy Danych}
            Zgodnie z deklaracjami w \emph{pom.xml} będziemy używać bazy danych H2. Jest to proste rozwiązanie w sam raz do celów deweloperskich. Taka baza danych jest włączana za każdym razem, gdy uruchamiamy nasz program i niszczona zaraz po jego zakończeniu. Z tego powodu nie musimy pamiętać o jej ręcznym instalowaniu, włczaniu, wyłączaniu i pielęgnowaniu. 
            
            Niestaty jest to też jej duża bolączka, ponieważ za każdym razem gdy wyłączymy nasz program tracimy całą zawartość bazy danych. Na szczeście można to łatwo naprawić - wystarczy podmienić domyślną konfigurację, na taką, która będzie przechowywała dane w plikach. Stracimy w ten sposób na szybkości działania, jednak nie będziemy już tracili danych przy restracie porgramu. \\
            Dodajmy więc do pliku \emph{application.properties} kilka linijek.
            \begin{minted}{properties}
spring.datasource.url=jdbc:h2:file:./db
spring.datasource.driverClassName=org.h2.Driver
spring.datasource.username=sa
spring.datasource.password=admin
spring.jpa.database-platform=org.hibernate.dialect.H2Dialect
spring.jpa.hibernate.ddl-auto=update
spring.h2.console.enabled=true
spring.h2.console.path=/h2
            \end{minted}
            Pierwszych sześć linijek to konfiguracja samego sterownika dla bazy danych. Tak żeby Hibernate wiedział gdzie jest baza, którą ma obsłużyć. Ciekawą rzeczą są dwie ostatnie linijki. Uruchamiają one wbudowany w H2 serwer, który w czasie działania aplikacji pozowli nam na podglądanie bazy danch. Internetową konsolę znajdziemy pod adresem \emph{http://localhost:8080/h2}
        
        \subsection{Obiekty DAO}
            \textbf{DAO} - Data Access Object 
            
            To właśnie dzięki nim będziemy mogli modelować naszą bazę. Zwyczajowo, każda klasa opisująca obiekt DAO, kończy się właśnie tym akronimem. 
            \subsubsection{CarDAO}
                Stwórzmy prosty obiekt opisujący samochód.
                \begin{minted}{java}
@Data
@Entity
@Table(name = "car")
public class CarDAO {
    @Id @GeneratedValue(strategy = GenerationType.IDENTITY)
    private Long id;

    private String numberPlate;
    private String seatsNumber;
    private Double totalDistance;
}
                \end{minted}
                \begin{description}
                \item[@Data] adnotacja Lomboka, która automatycznie w czasie kompilacji stworzy gettery, settery, funkcje konwertujące obiekt do stringu, funkcje prównujące obiekty między sobą i wiele innych
                \item[@Entity] wskazuje Springowi, że obiekt będzie trzeba zapisać w bazie danych
                \item[@Table] ustawia nazwę tabeli w bazie danych
                \item[@Id] wskazuje które z pól będzie kluczem głównym relacji
                \item[@GeneratedValue] definiuje w jaki sposób należ tworzyć wartości dla pola $id$. Zwuważmy, że taką adnotację można użyć do dowolnego pola, które będzie mogło być obsłużone przez jakiś skewenser. W naszym przypadku wskazujemy na chęć użycia sekwensera w taki sposób aby każda wartość $id$ była unikalna w całym cyklu życia tabeli   
                \end{description}
            \subsubsection{ClientDAO}
                Teraz twoim zadaniem będzie stworzenie obiektu klienta. Powinie nazywać się \textbf{ClientDAO} i mieć następujące pola:
                \begin{itemize}
                    \item name
                    \item surname
                    \item govID
                \end{itemize}
                Jeśli zrobiłeś wszystko dobrze to powinieneś otrzymać następujący kod
                \begin{minted}{java}
@Entity
@Data
@Table(name = "client")
public class ClientDAO {
    @Id @GeneratedValue(strategy = GenerationType.IDENTITY)
    private Long id;

    private String name;
    private String surname;
    private String govID;
}
            \end{minted}
        \subsubsection{Połączenie CarDAO i ClientDAO}
            W tym momencie mamy dwa osobne obiekty. CarDAO - reprezentujący auto i ClientDAO - reprezentujący klienta. Teraz należałoby stworzyć jakieś połączenie między nimi. Oczywiście takim połączeniem można zarządzać ręcznie, albo skorzystać z możliwości ORM. My skorzystamy z ORM. 
            
            Najpierw zastanówmy się jaki typ połączenia należało by wybrać. Zgodnie z logiką chcielibyśmy, aby jeden samochód był wyporzyczony tylko i wyłącznie przez jednego klienta w danym momencie. Natomiast nic nie broni, aby jeden klient wyporzyczał wiele aut naraz. Oznacza to połączenie \textbf{jeden} klient \textbf{do wielu} samochodów. 
            
            Dodajmy więc taką informację do obiektów DAO. W klasie CarDAO dopisujemy
            \begin{minted}{java}
@ManyToOne(
        fetch = FetchType.LAZY
)
private ClientDAO rentedBy;
            \end{minted}
            a w klasie ClientDAO
            \begin{minted}{java}
@OneToMany(
        mappedBy = "rentedBy",
        fetch = FetchType.LAZY
)
private Collection<CarDAO> rentedCars;
            \end{minted}
            Ciekawym elementem jest argument $fetch$. Definuje on  w jaki spodób będą pobierane dane z bazy. Zwróćmy uwagę, że zarówno obiekt klienta jak i obiekt samochodu posiadają referencje do siebie nawzajem. Tworzy to stosunkowo niebezpieczną sytuację, w której Hibernate próbowałby pobrać informacje o kliencie, w którym znalazłby referencję do samochdów jakie dany klient wyporzyczył, a w każdym samochodzie znalazłby referencję do osób go wyporzyczających, a w każdej osobie referencję do auto jakie wyporzyczyła i tak tworzymy nieskończoną rekurencję, przepełniamy stos w maszynie wirtualnej (JVM) i crashujemy program. Z tego powodu ustawamy argument $fetch$ na $LAZY$. Oznacza to, że ORM będzie pobierał informacje z bazy danych tylko i wyłącznie gdy program będzie potrzebował do nich dostępu. W ten sposób unikniemy nieskończonej rekurencji. Oczywiście wartość $LAZY$ jest ustawiana przez Spring domyślnie, aby przyspieszyć działanie aplikacji i nie ma potrzeby nadpisywania jej tą samą wartością, jednak zrobiliśmy to, żeby pokazać bardzo poważny problem na jaki niedługo natkniemy się jeszcze raz.
        \subsection{Dostęp do danych}
            Do tej pory zajmowaliśmy się tylko definiowaniem wyglądu bazy danych. Teraz przyjżymy się jak uzyskać dostęp do zgoromadzonych danych. W tym celu skorzystamy z kolejnego udogodnienia modułu Spring Data - czyli repozytoriów. 
            
            Są to interfejsy za pomocą, których będziemy definiowali polecenia dla bazy danych. Zauważmy, że w żadnym przypadku nie będziemy używali poleceń SQL. Pozwala to na maksymalną abstrakcję od tego jakiej bazy danych będziemy używali. Aktualnie jest to H2, ale nic nie szkodzi aby podpiąć do naszego projektu MySQL, OracleDB, Postgree, Aurore, albo RedShift.
            
            Zacznijmy od zdefiniowania Repozytorium dla klienta
            \begin{minted}{java}
public interface ClientRepository extends CrudRepository<ClientDAO, Long> {
    Boolean existsByGovID(String govID);
}
            \end{minted}
            a teraz dla pojazdu
            \begin{minted}{java}
public interface CarRepository extends CrudRepository<CarDAO, Long> {
    Boolean existsByNumberPlate(String numberPlate);
    Collection<CarDAO> findAllByRentedBy(ClientDAO clientDAO);
}
            \end{minted}
            Zauważmy, że nie piszemy całych repozytoriów samemu. Rozszerzamy już predefiniwane repozytorium - $CrudRepository$. Znajdują się w nim definicje popularnych poleceń, tworzenia, aktualizacji, usuwania i odczytu danych. Pamiętajmy też, że \textbf{repozytoria nie są klasami tylko interfejsami}. 
    \section{Logika aplikacji}
        W tym momencie mamy już wszystkie elementy potrzebne do stworzenia logiki naszej aplikacji. Zwyczajowo logikę umieścimy w specjalnych klasach - serwisach. Robi się tak z wielu powodów a niektóre z nich zobaczysz na przykładach w dalszych częściach tego tutorialu. 
        
        Stworzymy dwa serwisy, po jednym dla logiki klientów i aut. 
        \subsection{CarService}
        Zacznijmy od serwisu aut - \textbf{CarService}
        \begin{minted}{java}
@Service
public class CarService {
}
        \end{minted}
        Użyliśmy tutaj specjalnej adnotacji $@Service$. Podpowiada on Springowi, że klasa którą piszemy jest specjalnego przeznaczenia - jest serwisem. Spring w czasie uruchamiania aplikacji za każdym razem przegląda wszystkie klasy w celu znalezienia takiej adnotaji. 
        
        Jako że piszemy logikę dla samochodu powinniśmy mieć powiązanie z repozytorium odpowiedzialnym za interakcję z samochodami w bazie danych. Dodajmy je.
        \begin{minted}{java}
@Service
public class CarService {
    @Autowired
    private CarRepository carRepository;  
}

        \end{minted}
        
        Przyjżyj się adnotacji $@Autowired$. Dzięki niej Spring sam w czasie tworzenie obiektów klasy $CarService$ będzie inicjował zmienną $carRepository$ właściwymi wartościami. Po naszej stronie nie musimy już nic robić. Jest to jedna z zalet opisania naszej klasy jako $@Service$. Od teraz możemy korzystać z IoC - Inversion of Controll i DI - dependency injection. Co więcej, teraz nasza klasa serwisu też będzie mogła być wstrzykiwana w taki sam sposób jak repozytorium w każdej klasie zarządzanej przez IoC Springa. 
        
        Po zdefiniowaniu wszystkich potrzebnych zmiennych, możemy zacząć pisać właściwą część logiki. Zacznijmy od tworzenia nowych obiektów w bazie danych. Dodajmy funkcję za to odpowiedzialną.
        \begin{minted}{java}
public CarDAO create(CarDAO carDAO){
    return this.carRepository.save(carDAO);
}
        \end{minted}
        
        Niestety funkcja $CarRepository::save()$ nie potrafi sama określić czy dany obiekt już istnieje w bazie danych czy nie. Dzieje się tak dlatego, że wybraliśmy jako klucz główny sztucznie generowane $id$, a nie numer rejestracji. W momencie gdy wywołujemy tą funkcję, nie znamy jeszcze wartości $id$. W obiekcie CarDAO ustawialiśmy, że ta wartość ma być zarządzana przez bazę danych, więc zostanie stworzona dopiero po wywołaniu funkcji $CarRepository::save()$. Na szczęście nie jest to problem, skorzystamy z tego, że zawsze znamy numer rejestracji. Natomiast dzięki tworzeniu sztucznego $id$ mamy 100\% pewność, że żadne dwa $id$ się nie zdublują. Rozwiążmy teraz problem sprawdzania, czy dane auto istniej w bazie danych.
        \begin{minted}{java}
private Boolean checkIfExists(CarDAO carDAO){
    if(carDAO.getNumberPlate() == null) return false;
    return
            this.carRepository.existsByNumberPlate(carDAO.getNumberPlate());
}
        \end{minted}
        
        Na początku sprawdzamy, czy dane auto ma wogóle rejestrację, jak jej nie ma to i tak już nie mamy szans na jego rozpoznanie. Później korzystam z funkcji, którą zdefiniowaliśmy w naszym repozytorium do sprawdzania czy istniją obiekty w bazie danych o podanej rejestracji.
        
        Mając obie funkcjie nareszcie możemy stworzyć rozwiązanie, które nie będzie niszczyło naszej bazy przez przypadek, czyli:
        \begin{minted}{java}
public CarDAO createIfNotExits(CarDAO carDAO){
    if(this.checkIfExists(carDAO))
        return carDAO;
    else
        return this.create(carDAO);
}
        \end{minted}
        
        Z punktu widzenia naszej aplikacji potrzebne są nam jeszcze trzy funkcjonalności - towrzenie nowych aut, wyszukiwanie już istniejących i zarządzanie ich wyporzyczaniem. Zaimplementujmy teraz funkcję odpowiedzialną za wyszukiwanie w bazie danych już istniejących aut.
        \begin{minted}{java}
public CarDAO getOne(Long id) throws Exception {
    Optional<CarDAO> optionalCarDAO =
            this.carRepository.findById(id);
    if(optionalCarDAO.isEmpty())
        throw new ProcessingException("Car not found");
    else return optionalCarDAO.get();
}
        \end{minted}
        
        Używając funkcji $CarRepository::find()$ otzymujemy specjalny obiekt - $Optional$. Jest to wrapper na dowolny obiekt, który dodatkowo przetrzymuje informację, czy obiekt w nim się znajdujący to przypadkiem nie null. Użyliśmy także specjalnie stworzonego wyjątku - $ProcessingException$. Pomoże nam to w przyszłości tłumaczyć wyjątki rzucane przez Spring do odpowiedzi API. Odejdźmy na chwilę od naszej klasy serwisu i napiszmy właśnie użytą klasę $ProcessingException$
        \begin{minted}{java}
public class ProcessingException extends Exception{
    public ProcessingException(String message){
        super(message);
    }
}
        \end{minted}
        
        Teraz wróćmy spowrotem do pisanego serwisu i dodajmy dwie ostatnie funkcje - wyporzyczanie samochodu i jego oddawania. Pamiętajmy, że w przechowujemy inforację o całkowitym przebytym dystansie przez każdy samochód i należy go aktualizować w przy każdym oddaniu auta. Zwróćmy też uwagę na zapobieganie oddawania cudzego auta, lub wyporzyczania samochodu już wyporzyczonego.
        \begin{minted}{java}
public CarDAO rent(CarDAO carDAO, ClientDAO clientDAO) throws Exception {
    if (carDAO.getRentedBy() == null){
        carDAO.setRentedBy(clientDAO);
        return this.carRepository.save(carDAO);
    }
    if (carDAO.getRentedBy().equals(clientDAO))
        throw new ProcessingException("Already Rented By You");
    if (carDAO.getRentedBy() != null)
        throw new ProcessingException("Already Rented");
    return null;
}

public CarDAO returnn(CarDAO carDAO, ClientDAO clientDAO, Double distanceCovered) throws Exception {
    if (carDAO.getRentedBy() == null)
        throw new ProcessingException("Car is not rented by anyone");
    if (!carDAO.getRentedBy().equals(clientDAO))
        throw new ProcessingException("Car is not rented by you");
    carDAO.setRentedBy(null);
    carDAO.setTotalDistance(
            carDAO.getTotalDistance() + distanceCovered
    );
        eturn this.carRepository.save(carDAO);
}
        \end{minted}
        W tym momencie skończyliśmy pisanie serwisu odpowiedzialnego za logikę aut.
        \subsection{ClientService}
        Teraz twoim zadaniem jest napisanie podobnego serwisu, ale dla klienta. Wymagane funkcjonalności to:
        \begin{itemize}
            \item dodawanie nowych klientów - $createIfNotExits$
            \item szukanie jednego klienta - $getOne$
            \item wyporzyczanie i oddawanie samochodów - zwróć uwagę, że argumentami funkcji powinny być id auta i klienta + przebyty dystans
        \end{itemize}
        Jeżeli napisałeś wszystko dobrze to twój kod powinien być podobny do tego:
        \begin{minted}{java}
@Service
public class ClientService {
    @Autowired
    private CarService carService;

    @Autowired
    private ClientRepository clientRepository;

    private ClientDAO create(ClientDAO clientDAO){
        if (clientDAO.getRentedCars() == null)
            clientDAO.setRentedCars(new ArrayList<>());
        return this.clientRepository.save(clientDAO);
    }

    private Boolean checkIfExists(ClientDAO clientDAO){
        if (clientDAO.getGovID() == null) return false;
        return this.clientRepository.existsByGovID(clientDAO.getGovID());
    }

    private Boolean checkIfExist(Long id){
        return this.clientRepository.existsById(id);
    }

    public ClientDAO createIfNotExist(ClientDAO clientDAO){
        if (this.checkIfExists(clientDAO))
            return clientDAO;
        else
            return this.create(clientDAO);
    }

    public ClientDAO getOne(Long id) throws Exception {
        Optional<ClientDAO> clientDAOOptional =
                this.clientRepository.findById(id);
        if (clientDAOOptional.isEmpty())
            throw new ProcessingException("Not Found");
        return clientDAOOptional.get();
    }

    public Collection<ClientDAO> findAll(){
        return (Collection<ClientDAO>) this.clientRepository.findAll();
    }

    public ClientDAO rent(CarDAO carDAO, ClientDAO clientDAO) throws Exception {
        this.carService.rent(carDAO, clientDAO);
        return clientDAO;
    }

    public ClientDAO rent(Long carId, Long clientId) throws Exception {
        return this.rent(
                this.carService.getOne(carId),
                this.getOne(clientId)
        );
    }

    public ClientDAO returnn(CarDAO carDAO, ClientDAO clientDAO, Double distanceCovered) throws Exception {
        this.carService.returnn(carDAO, clientDAO, distanceCovered);
        return clientDAO;
    }

    public ClientDAO returnn(Long carId, Long clientId, Double distance) throws Exception {
        return this.returnn(
                this.carService.getOne(carId),
                this.getOne(clientId),
                distance
        );
    }
}
        \end{minted}

    \section{Warstwa Prezentacji}
    
        Ostatnią rzeczą do wykonania zostaje część programu odpowiedzialna za komunikację z użytkownikiem. Jest to warstwa prezentacji. Zwyczajowo klasy odpowiedzialne za komunikowanie się z użytkownikiem to kontrolery. Z tego powodu będa posiadały końcówkę $Controller$.
        \subsection{Obiekty Transferowe}
            Zanim przejdziemy do implementacji kontolerów, musimy zająć się problemem nieskończonej rekurencji w obiektach DAO. Dla warstwy logiki biznesowej naprawiliśmy ten problem stosując \emph{lazy-fetching}. Oznaczało to tyle, że ORM nie próbował wyciągać z bazy danych całego obiektu a tylko te części, które były w danym momencie potrzebne. 
            
            Teraz problem jest o wiele poważniejszy. Jeżeli nic z nim nie zrobimy to pojawi się w momencie serializowania obiektu do formatu JSON - w prostych słowach, gdzy będziemy chcieli pokazać coś użytkownikowi to zabijemy go niekończonym strumieniem danych - nie dobrze. Stanie się tak, ponieważ moduł odpowiedzialny za zamienianie obiektów do sformatowanego do postaci JSONa tekstu, będzie starał się przetrawersować CAŁY obiekt. To oznacza, że zmusi ORM do pobierania coraz głębszych warstw obiektu, a to daje skutek podobny do natychmiastowego pobrania całego obiektu do pamięci RAM - bardzo źle. 
            
            Rozwiązania tego problemu są dwa. Możemy pobawić się adnotacjami i wskazać Jacksonowi co należy serializować a co lepiej sobie odupścić i w ten sposób przerwiemy cykl. Alternatywnie możemy skorzystać z obiektów transferowych - DTO - Data Transfer Object. 
            
            DTO najczęściej zawierają tylko pewne pola z DAO, w taki sposób aby zapobiegać zapętleniom. Co więcej, jedeo DAO może posiadać wiele DTO. Jest to największa przewaga wykorzystania obiektów transferowych nad sterowaniem serializacją za pomocą adnotacji.
            
            \subsubsection{CarDTO}
                Zacznijmy od stworzenia obiektu transferowego dla auta. Kierując się logiką stwierdzamy, że użytkownik samochodu, nie musi wiedzieć, że samochód, który wyporzycza jest wyporzyczany przez niego. Skoro go wyporzyczył to już to wie. Z tego powodu usuniemy z obiektu samochodu informację o tym kto go wyporzycza, a przez to przerwiemy cykl.
                \begin{minted}{java}
@Data
@Builder
@AllArgsConstructor
public class CarDTO {
    private Long id;
    private String numberPlate;
    private String seatsNumber;
    private Double totalDistance;

    public static CarDTO toDTO(CarDAO entity){
        return CarDTO.builder()
                .id(entity.getId())
                .numberPlate(entity.getNumberPlate())
                .seatsNumber(entity.getSeatsNumber())
                .totalDistance(entity.getTotalDistance())
                .build();
    }
}
                \end{minted}
                \begin{description}
                    \item[@Builder] jest adnotacją Lomboka, która pozwala na wykorzystanie wzorcu budowniczego (Builder Pattern) do tworzenia obiektów danej klasy
                    \item[@AllArgsConstructor] kolejna adnotacja Lomboka, pozwalająca na stworzenie konstruktora skladającego się ze wszystkich pól danej klasy. Domyślnie adnotacja $@Data$ zawiera pusty konstruktor - taki, który nie posiada argumentów
                \end{description}
            \subsubsection{ClientDTO}
                Teraz, czas na twoją pracę. W podobny sposób zaimplementuj obiekt transferowy dla klienta. Pamiętaj o tym, że  $ClientDAO$ w polu $rentedCars$ przechowuje całą listę obiektów $CarDAO$ do których przed chwilą stworzyliśmy obiekty transferowe. Pamiętaj, że w obiekcie transferowym nie można przechowywać obiektów DAO, dlatego musisz zamienić je na DTO.
                
                O ile zrobiłeś wszystko dobrze to powinieneś otrzymać kod podobny do tego:
                \begin{minted}{java}
@Data
@Builder
@AllArgsConstructor
public class ClientDTO {
    private Long id;
    private String name;
    private String surname;
    private String govID;
    private LocalDateTime lastSeen;
    private Collection<CarDTO> rentedCars;

    public static ClientDTO toDTO(ClientDAO entity){
        return ClientDTO.builder()
                .id(entity.getId())
                .name(entity.getName())
                .surname(entity.getSurname())
                .govID(entity.getGovID())
                .lastSeen(entity.getLastSeen())
                .rentedCars(
                        entity.getRentedCars().stream()
                                .map(CarDTO::toDTO)
                                .collect(Collectors.toList())
                )
                .build();
    }
}
                \end{minted}
        \subsection{Kontrolery}
            Jeżeli rozwiązaliśmy problem z obiektami DTO i DAO, to możemy przejść do już ostatniej części tego laboratorium. Stworzymy teraz kontrolery.
            \subsubsection{CarController}
                Bedzie to obiekt odpowiedzialny za interakcję użytkownika z samochodem. Chcielibyśmy umożliwić klientowi dwie operacje:
                \begin{description}
                    \item[GET /car/\{id\}] pozwoli na odczytywanie informacji o samochodzie 
                    \item[POST /car/]  pozwoli na dodawanie nowych samochodów
                \end{description}
                To czym jest $GET$ i $POST$ oraz jakie są między nimi różnice omówiliśmy na prezentacji do tego laboratorium. Jednak jeżeli chciałbyś się dowiedzieć więcej zapraszam do linków na końcu skryptu.
                
                Zacznijmy od zdefiniowania odpowiedniej klasy dla naszego kontrolera
                \begin{minted}{java}
@RestController
@RequestMapping("/car")
public class CarController { 
    @Autowired
    private CarService carService;
}
                \end{minted}
                \begin{description}
                    \item[@RestController] to tak naprawdę złożenie dwóch adnotacji. @Controller - która mówi o tym, że pisana właśnie klasa ma specjalne przeznaczenie, będzie kontrolerem. Drugą adnotacją jest @ResponseBody, wskazuje ona na to, że zwracana przez funkcję obiekty mają być serializowane do postacji JSON
                    \item[@RequestMapping] wskazuje na to jakie rządania mają być obsługiwane przez ten kontroler. Tutaj wszystko co zaczyna się od \emph{/car}
                \end{description}
                Teraz dodamy nasz pierwszy endpoint - \emph{GET /car/\{id\}}
                \begin{minted}{java}
@GetMapping("/{id}")
public CarDTO getOne(
        @PathVariable("id") Long id
) throws Exception {
    return CarDTO.toDTO(
            this.carService.getOne(id)
    );
}                   
                \end{minted}
                \begin{description}
                    \item[@GetMapping] wskazuje, że ten endpoint będzie odbierał tylko rządania typu $GET$, kończące się na $/\{id\}$
                    \item[@PathVariable] wskazuje, że do parametru id należy wpisać wartość, która znajdzie w miejscu \emph{\{id\}}  
                \end{description}
                
                Jak widzimy jest to prosty endpoint, który wywoła funkcję $CarService::getOne$ i przekonwertuje jej wynik do obiektu DTO
                
                Teraz dodamy endpoint odpowiedzialny za dodawanie nowgo auta
                \begin{minted}{java}
@PostMapping("/")
public CarDTO postOne(
        @RequestBody CarDAO carDAO
){
    return CarDTO.toDTO(
            this.carService.createIfNotExits(carDAO)
    );
}
                \end{minted}
                
                Nowością tutaj jest wykorzystanie adnotacji $@RequestBody$. Pozwala ona na automatyczną deserializację obiektu zawartego w rządaniu do obiektu Javy. Reszta jest identyczna.
            \subsubsection{ClientController}
                W przypadku kontolera klienta chcielibyśmy udostępnić następujące endpointy:
                \begin{description}
                    \item[GET /client/\{id\}] zwróci informacje o kliencie i wyporzyczonych przez niego samochodach
                    \item[POST /client/] doda nowego klienta
                    \item[PATCH /client/rent/\{id\}] pozwoli wyporzyczyć samochód
                    \item[PATCH /client/return/\{id\}] pozwoli oddać wypożyczony samochód
                \end{description}
                
                Podstawowe funkcjonalności tego endpointu są dokładnie takie same, więc pominiemy ich dokładniejszy opis.
                \begin{minted}{java}
@RestController
@RequestMapping("/client")
public class ClientController {
    @Autowired
    private ClientService clientService;

    @GetMapping("/{id}")
    public ClientDTO getOne(
            @PathVariable("id") Long id
    ) throws Exception {
        return ClientDTO.toDTO(
                this.clientService.getOne(id)
        );
    }

    @PostMapping("/")
    public ClientDTO postOne(
            @RequestBody ClientDAO clientDAO
    ){
        return ClientDTO.toDTO(
                this.clientService.createIfNotExist(clientDAO)
        );
    }
}
                \end{minted}
                
                O tym czym jest rządanie typu $PATCH$ także mówiliśmy na prezentacji, więce tak jak poprzednio, zainteresowanych odsyłam do linków na końcu skryptu.
                
                Zakończymy teraz proces tworzenia kontrolera dodając jesgo ostatnie dwa endpointy typu $PATCH$
                \begin{minted}{java}
@PatchMapping("/rent/{car_id}")
public ClientDTO retnOne(
        @PathVariable("car_id") Long carId,
        @RequestParam("client_id") Long clientId
) throws Exception {
    return ClientDTO.toDTO(
            this.clientService.rent(carId, clientId)
    );
}

@PatchMapping("/return/{car_id}")
public ClientDTO returnOne(
        @PathVariable("car_id") Long carId,
        @RequestParam("client_id") Long clientId,
        @RequestParam("distance") Double distance
) throws Exception {
    return ClientDTO.toDTO(
            this.clientService.returnn(
                    carId, clientId, distance)
    );
}
                \end{minted}
                \begin{description}
                    \item[@PatchMapping] jest to adnotacja podobna do $@GetMapping$ 
                    \item[@RequestParam] w przeciwieństwie do adnotacji $@PathVariable$, która szuka danych tylko w linku i $@RequestBody$, która próbuje deserializować całą zawartość rządania do jednego obiektu $@RequestParam$ szuka w rządaniu jednego konkretnego klucza i wyciąga z niego wartość. Nie ważne gdzie taki klucz się znajduje. 
                \end{description}
    \section{Podsumowanie}
        Właśnie stworzyłeś działający i zgodny z większością standardów serwer. Dodatkowo zrobiłeś to w sposób w jaki robi się to podczas zawodowego tworzenia takich rozwiązań. Pomimo tego, że wiele elementów zostało tutaj uproszczonych, po to żeby nie przedłużać i tak już długiego skryptu, to stworzyłeś pełnoprawny serwer backendowy. Oczywiście brakuje w nim wielu elementów, np. poprawnej obsługi błędów, dodania standardu HATEOAS, odpowiednich zabezpieczeń, systemu autoryzacji, itd. jednak główny celem tego laboratorium było pokazanie podstawowych funkcjonalności frameworku Spring.
        
        W czasie tego laboratorium nauczyłeś się:
        \begin{itemize}
            \item Jak współcześnie modelujemy i zarządzamy bazą danych z poziomu kodu
            \item Czym jest DAO i jakie wiążą się z nim problemy
            \item Do czego używamy DTO i czym jest kontroler
        \end{itemize}
    \section{tl;dr}
        Linki dla chętnych
        \begin{itemize}
            \item \href{https://en.wikipedia.org/wiki/Model%E2%80%93view%E2%80%93controller}{Co to MVC}
            \item \href{https://developer.mozilla.org/en-US/docs/Web/HTTP/Methods}{Typy rządań HTTP}
            \item \href{https://github.com/ptylczynski/BAI-labo-sketch}{Wygenerowany projekt Initializr}
        \end{itemize}
\end{document}